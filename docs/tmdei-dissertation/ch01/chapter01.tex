% Chapter 1
% 
\chapter{Introdução} % Main chapter title
\label{chap:Chapter01} % For referencing the chapter elsewhere, use Chapter~\ref{Chapter01}
%-------------------------------------------------------------------------------
%---------
%

\section{Contexto} 
\label{sec:chap01_context} %For referencing this section elsewhere, use Section~\ref{sec:chap1_introduction}

A alfabetização em Braille é um fator de extrema importância para a independência e inclusão de pessoas com deficiência visual \parencite{REF01}. O sistema Braille, inventado por Louis Braille no século XIX \parencite{REF02}, é amplamente utilizado em todo o mundo como método de leitura e escrita para cegos, permitindo que essas pessoas acessem informação e comunicação de forma eficaz.

Nos últimos anos, tem havido um crescente reconhecimento da importância da acessibilidade e inclusão para pessoas com deficiência visual. O aprendizado do Braille e o acesso a ferramentas como a máquina de escrever em Braille desempenham um papel fundamental na promoção da independência e autonomia dessas pessoas, não apenas na educação, mas também no trabalho e na vida cotidiana. No entanto, a aprendizagem do Braille não é isenta de desafios. O elevado custo das máquinas Braille e a escassez de unidades disponíveis em muitos contextos educacionais ou de reabilitação dificultam o acesso e a prática, criando barreiras para o aprendizado \parencite{REF01}.

Em ambientes educacionais e de reabilitação, onde o ensino do Braille é fundamental, a necessidade de múltiplas máquinas de escrever para atender a vários alunos representa uma barreira significativa, tanto em termos de recursos financeiros quanto de infraestrutura. A limitação no número de máquinas Braille em instituições de ensino ou reabilitação pode restringir a prática individual dos alunos, prejudicando o ritmo de aprendizado e a autonomia dos estudantes.

Nesse cenário, surge a necessidade do desenvolvimento de ferramentas acessíveis e inovadoras que permitam a prática do Braille de forma mais econômica e eficiente e que facilitem o acesso às máquinas de escrever em Braille. É nesse contexto que se insere este trabalho, que propõe o desenvolvimento de uma aplicação web e gratuita, que simula o funcionamento da máquina de escrever em Braille.

A aplicação proposta visa superar algumas das limitações das máquinas de escrever em Braille tradicionais, oferecendo uma alternativa acessível, prática e eficaz para o aprendizado e prática do Braille. Simulando as funcionalidades de uma máquina de escrever em Braille através do teclado do computador, a aplicação permite que os usuários pratiquem o Braille de forma virtual e interativa, facilitando a transição para o uso real da máquina Braille.

Além disso, a aplicação oferece a visualização do texto em tinta, tornando o conteúdo acessível para usuários que não estão familiarizados com o Braille. Com isso, o trabalho proposto contribui não apenas para a melhoria da acessibilidade e inclusão de pessoas com deficiência visual, mas também para a promoção do aprendizado e uso do Braille em um contexto mais amplo.

\section{Problema}

O aprendizado do Braille e o acesso às máquinas de escrever em Braille representam desafios significativos para pessoas com deficiência visual, especialmente para aqueles que ficaram cegos recentemente. A escassez de recursos acessíveis e o valor elevado das Máquinas Braille dificultam o acesso a ferramentas adequadas para o aprendizado e prática do Braille \parencite{REF01}. Além disso, a necessidade de múltiplas máquinas de escrever em Braille para práticas em grupo, como em ambientes educacionais, representa uma barreira adicional para o acesso e a inclusão de pessoas com deficiência visual.

Assim, surge a necessidade de desenvolver soluções inovadoras e acessíveis que facilitem o aprendizado do Braille e o acesso às máquinas de escrever em Braille de forma mais eficaz e inclusiva. A necessidade de uma solução que permita a prática virtual, de baixo custo e acessível, é o ponto central abordado por esta dissertação, que visa contribuir com a criação de um ambiente inclusivo para o aprendizado e a prática do Braille, facilitando o acesso de pessoas com deficiência visual a essas ferramentas fundamentais.

Diante desse problema, surge a seguinte questão: Como desenvolver uma aplicação web simuladora da máquina de escrever em Braille que seja acessível, prática e eficaz para o aprendizado e prática do Braille, contribuindo assim para a promoção da inclusão e autonomia de pessoas com deficiência visual?




\section{Objetivos}

\subsection{Objetivo Geral}

Desenvolver uma aplicação web gratuita que simule a máquina de escrever em Braille, oferecendo uma ferramenta acessível, prática e eficaz para o aprendizado e prática do sistema Braille. O projeto visa promover a inclusão e autonomia de pessoas com deficiência visual, facilitando o acesso ao ensino e à prática do Braille, especialmente para aqueles que enfrentam barreiras financeiras e tecnológicas ao utilizar máquinas Braille físicas.

\subsection{Objetivos específicos}

\begin{itemize}
    \item Projetar e implementar uma interface acessível e intuitiva, assegurando usabilidade para pessoas com deficiência visual, com uma experiência fluida e amigável para todos os usuários, seguindo boas práticas de acessibilidade web.

    \item Desenvolver algoritmos que simulem as funcionalidades da máquina de escrever em Braille, mapeando as teclas do teclado do computador para os pontos Braille correspondentes, permitindo uma experiência de digitação próxima com a real.

    \item Implementar o Modo Livre, que permite aos usuários praticarem a escrita em Braille de forma livre, com a visualização das células Braille correspondentes diretamente na tela.

    \item Desenvolver o Modo Desafio, no qual são apresentadas palavras para que o usuário pratique digitando-as corretamente, com feedback imediato, promovendo o aprendizado de forma interativa e envolvente.

    \item Integrar funcionalidades adicionais, como a visualização do texto em tinta, para facilitar o uso por pessoas que não são familiarizadas com o sistema Braille, ampliando o alcance e a acessibilidade da aplicação.

    \item Realizar testes e avaliações com usuários com deficiência visual, assegurando que a aplicação atenda aos critérios de eficácia, acessibilidade e usabilidade, permitindo ajustes e melhorias baseados em feedback real.

    \item Avaliar o impacto da aplicação no aprendizado e prática do Braille, medindo como ela contribui para a promoção da inclusão e autonomia de pessoas com deficiência visual, e identificando melhorias contínuas para maximizar esses benefícios.

\end{itemize}

Atingindo esses objetivos, espera-se que a aplicação desenvolvida neste trabalho se torne uma ferramenta valiosa para o aprendizado e prática do sistema Braille. Com isso, o projeto contribuirá para a inclusão social e para a promoção da autonomia de pessoas com deficiência visual, oferecendo uma solução acessível e eficaz.

\section{Questões de Investigação}

Para orientar o desenvolvimento deste trabalho e garantir que os objetivos propostos sejam alcançados, é necessário levantar questões-chave que guiarão o processo de pesquisa. Essas questões de investigação buscam explorar os desafios e oportunidades relacionados ao aprendizado e à prática do sistema Braille por meio de uma aplicação web que simule uma máquina de escrever em Braille. Elas também visam compreender como a o projeto pode contribuir para a inclusão e a autonomia de pessoas com deficiência visual.

\begin{table}[h]
    \caption{Questão global}
    \label{tab:ch01-global-question}
    \centering
    \begin{tabular}{c>{\raggedright\arraybackslash}p{0.8\linewidth}}
        \toprule
        \tabhead{Questão}& \tabhead{Questão de Investigação}\\
        \midrule
        Qg& Como a aplicação pode contribuir para democratizar o aprendizado do Braille e da máquina de escrever Braille, promovendo autonomia a pessoas com deficiência visual?\\
        \bottomrule\\
    \end{tabular}
\end{table}

A questão global, descrita na tabela \ref{tab:ch01-global-question}, norteia toda a pesquisa e busca compreender o impacto que a aplicação pode ter na democratização do uso da Máquina Braille e na melhoria do ensino de Braille a pessoas com ou sem deficiência visual. A investigação dessa questão permitirá avaliar como a ferramenta pode tornar o aprendizado do Braille mais acessível e eficiente, especialmente para aqueles que enfrentam barreiras financeiras e tecnológicas. A resposta a essa questão será determinante para avaliar a relevância e o impacto do estudo, confirmando sua contribuição para a inclusão social e a promoção da independência de seus usuários.

Além da questão global que orienta o desenvolvimento deste trabalho, foram definidas três questões específicas que visam detalhar aspectos fundamentais da pesquisa. Essas questões buscam avaliar a eficácia, as funcionalidades e o impacto da aplicação no aprendizado do Braille e na promoção da inclusão de pessoas com deficiência visual.

\begin{table}[h]
    \caption{Questões específicas}
    \label{tab:ch01-specific-questions}
    \centering
    \begin{tabular}{c>{\raggedright\arraybackslash}p{0.8\linewidth}}
        \toprule
        \tabhead{Questão}& \tabhead{Questão de Investigação}\\
        \midrule
        Qe1& Quais as características e funcionalidades que a tornam uma alternativa mais acessível e eficaz para o aprendizado da máquina Braille?\\
        \addlinespace
        Qe2& A aplicação é eficaz no ensino do Braille e da máquina de escrever Braille para usuários com diferentes níveis de conhecimento?\\
        \addlinespace
        Qe3& Qual é o impacto da aplicação no aprendizado e prática do Braille, bem como na promoção da inclusão e autonomia de pessoas com deficiência visual, e como esse impacto pode ser avaliado e quantificado?\\
        \bottomrule\\
    \end{tabular}
\end{table}

A Qe1 foca em identificar os elementos que fazem da aplicação uma solução viável em termos de acessibilidade e eficiência. Nessa questão, a investigação busca entender quais funcionalidades da simulação da máquina de escrever em Braille proporcionam uma experiência de aprendizado adequada, além de explorar os fatores que tornam essa solução mais econômica e prática em comparação com as máquinas Braille físicas. A resposta a esta questão irá fornecer informações sobre o potencial da aplicação para democratizar o acesso ao Braille. 

A Qe2 aborda a eficácia pedagógica da aplicação. Será analisado como a ferramenta atende tanto a iniciantes quanto a usuários mais avançados no aprendizado do Braille. A investigação nessa questão visa identificar se a aplicação pode ser usada como uma ferramenta inclusiva e flexível, capaz de adaptar-se a diferentes perfis de usuários e seus respectivos níveis de conhecimento. Ao investigar como a ferramenta pode se adaptar a diversos perfis de aprendizado, essa questão permitirá entender o alcance e a versatilidade da aplicação como um recurso educacional inclusivo.

Por fim, a Qe3 trata de medir o impacto real da aplicação no processo de aprendizado e inclusão. Serão analisados os benefícios gerados em termos de autonomia e qualidade de ensino para as pessoas com deficiência visual, bem como o efeito no aprendizado do Braille. A pesquisa também se concentrará em desenvolver formas de avaliação e quantificação desse impacto, considerando métricas como progresso no aprendizado, feedback de usuários e melhorias na capacidade de utilizar a máquina Braille real.
