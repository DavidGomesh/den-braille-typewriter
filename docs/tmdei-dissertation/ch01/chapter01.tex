% Chapter 1
% 
\chapter{Introdução} % Main chapter title
\label{chap:Chapter01} % For referencing the chapter elsewhere, use Chapter~\ref{Chapter01}
%-------------------------------------------------------------------------------
%---------
%

\section{Contexto} 
\label{sec:chap01_context} %For referencing this section elsewhere, use Section~\ref{sec:chap1_introduction}

Nos últimos anos, tem havido um crescente reconhecimento da importância da acessibilidade e inclusão para pessoas com deficiência visual. Nesse contexto, o aprendizado do Braille e o acesso a ferramentas como a máquina de escrever em Braille desempenham um papel fundamental na promoção da independência e autonomia dessas pessoas. No entanto, o acesso a recursos adequados para aprendizado e prática do Braille muitas vezes tem sido limitado devido à escassez de ferramentas acessíveis e ao alto custo das máquinas \parencite{REF01}.

Nesse cenário, surge a necessidade de desenvolver soluções inovadoras e acessíveis que facilitem o aprendizado do Braille e o acesso às máquinas de escrever em Braille. É nesse contexto que se insere este trabalho, que propõe o desenvolvimento de uma aplicação web gratuita simuladora da Máquina de escrever em Braille.

A aplicação proposta visa superar as limitações das máquinas de escrever em Braille tradicionais, oferecendo uma alternativa acessível, prática e eficaz para o aprendizado e prática do Braille. Ao simular as funcionalidades de uma máquina de escrever em Braille através do teclado do computador, a aplicação permite que os usuários pratiquem o Braille de forma virtual e interativa, facilitando a transição para o uso real da máquina de escrever em Braille.

Além disso, a aplicação oferece recursos adicionais, como a visualização do texto em tinta, tornando o conteúdo acessível para usuários que não estão familiarizados com o Braille. Com isso, o trabalho proposto contribui não apenas para a melhoria da acessibilidade e inclusão de pessoas com deficiência visual, mas também para a promoção do aprendizado e uso do Braille em um contexto mais amplo.

Em um momento em que a tecnologia desempenha um papel cada vez mais importante na vida cotidiana, esta dissertação busca explorar o potencial da tecnologia digital para promover a acessibilidade e inclusão, capacitando as pessoas com deficiência visual a alcançar seu pleno potencial através do aprendizado e uso do Braille.

\section{Problema}

O aprendizado do Braille e o acesso às máquinas de escrever em Braille representam desafios significativos para pessoas com deficiência visual, especialmente para aqueles que ficaram cegos recentemente. A escassez de recursos acessíveis e o valor elevado das Máquinas Braille dificultam o acesso a ferramentas adequadas para o aprendizado e prática do Braille \parencite{REF01}. Além disso, a necessidade de múltiplas máquinas de escrever em Braille para práticas em grupo, como em ambientes educacionais, representa uma barreira adicional para o acesso e a inclusão de pessoas com deficiência visual.

Diante desse contexto, surge a necessidade de desenvolver soluções inovadoras e acessíveis que facilitem o aprendizado do Braille e o acesso às máquinas de escrever em Braille de forma mais eficaz e inclusiva. Portanto, o problema central abordado por esta dissertação de mestrado é a falta de recursos acessíveis e eficazes para o aprendizado e prática da Máquina Braille, incluindo a escassez de máquinas de escrever em Braille acessíveis. Diante desse problema, surge a seguinte questão: Como desenvolver uma aplicação web simuladora da máquina de escrever em Braille que seja acessível, prática e eficaz para o aprendizado e prática do Braille, contribuindo assim para a promoção da inclusão e autonomia de pessoas com deficiência visual?


\section{Objetivos}

\subsection{Objetivo Geral}

Desenvolver uma aplicação web gratuita simuladora da máquina de escrever em Braille que proporcione uma solução acessível, prática e eficaz para o aprendizado e prática do sistema Braille, contribuindo para a promoção da inclusão e autonomia de pessoas com deficiência visual.

\subsection{Objetivos específicos}

\begin{itemize}
    \item Analisar as necessidades e requisitos dos usuários com deficiência visual para o desenvolvimento da aplicação web simuladora de máquina de escrever em Braille.

    \item Projetar e implementar a interface da aplicação, garantindo acessibilidade e usabilidade para usuários com deficiência visual, bem como uma experiência intuitiva e amigável para todos os usuários.

    \item Desenvolver os algoritmos necessários para simular as funcionalidades de uma máquina de escrever em Braille, mapeando as teclas do teclado do computador para os pontos Braille correspondentes.

    \item Implementar o Modo Livre da aplicação, permitindo aos usuários escreverem livremente em Braille e visualizarem as células Braille correspondentes na tela.

    \item Desenvolver o Modo de Lições da aplicação, oferecendo exercícios estruturados para guiar os usuários no aprendizado progressivo do sistema Braille.

    \item Integrar funcionalidades adicionais, como a visualização do texto em tinta, para facilitar o acesso ao conteúdo para usuários que não estão familiarizados com o Braille.

    \item Realizar testes e avaliações da aplicação com usuários com deficiência visual para garantir sua eficácia, acessibilidade e usabilidade.

    \item Avaliar o impacto da aplicação no aprendizado e prática do Braille, bem como na promoção da inclusão e autonomia de pessoas com deficiência visual.

\end{itemize}

Ao alcançar esses objetivos, espera-se que a aplicação web simuladora de máquina de escrever em Braille desenvolvida neste trabalho possa proporcionar uma ferramenta valiosa para o aprendizado e prática do Braille, contribuindo para a promoção da inclusão e autonomia de pessoas com deficiência visual.

\section{Questões de Investigação}

O trabalho tem como objetivo desenvolver uma aplicação web para simular uma máquina de escrever em Braille, visando tornar o aprendizado e prática da máquina Braille mais acessível e eficaz. Nesse contexto, são formuladas questões de pesquisa, que orientarão o desenvolvimento, implementação e avaliação da aplicação.

\begin{table}[h]
    \caption{Questão global}
    \label{tab:ch01-global-question}
    \centering
    \begin{tabular}{c>{\raggedright\arraybackslash}p{0.8\linewidth}}
        \toprule
        \tabhead{Questão}& \tabhead{Questão de Investigação}\\
        \midrule
        Qg& Como a aplicação pode contribuir para democratizar o aprendizado do Braille e da máquina de escrever Braille, promovendo autonomia a pessoas com deficiência visual?\\
        \bottomrule\\
    \end{tabular}
\end{table}

Esta questão global norteia toda a pesquisa e busca compreender o impacto da aplicação web na democratização da Máquina Braille, na autonomia e na qualidade de vida das pessoas com deficiência visual. A resposta a esta questão irá determinar a relevância e o impacto do estudo.

\begin{table}[h]
    \caption{Questões específicas}
    \label{tab:ch01-specific-questions}
    \centering
    \begin{tabular}{c>{\raggedright\arraybackslash}p{0.8\linewidth}}
        \toprule
        \tabhead{Questão}& \tabhead{Questão de Investigação}\\
        \midrule
        Qe1& Quais as características e funcionalidades que a tornam uma alternativa mais acessível e eficaz para o aprendizado da máquina Braille?\\
        \addlinespace
        Qe2& A aplicação é eficaz no ensino do Braille e da máquina de escrever Braille para usuários com diferentes níveis de conhecimento?\\
        \addlinespace
        Qe3& Qual é o impacto da aplicação no aprendizado e prática do Braille, bem como na promoção da inclusão e autonomia de pessoas com deficiência visual, e como esse impacto pode ser avaliado e quantificado?\\
        \bottomrule\\
    \end{tabular}
\end{table}

A Qe1 busca identificar as características e funcionalidades da aplicação que a distinguem da máquina de escrever Braille física e a tornam mais acessível e eficaz para o aprendizado. A resposta a esta questão irá fornecer informações sobre o potencial da aplicação para democratizar o acesso ao Braille. 

A Qe2 avalia a efetividade da aplicação em diferentes níveis de conhecimento do Braille, desde iniciantes até usuários experientes. 

Por fim, a Qe3 visa avaliar o impacto da aplicação no aprendizado e prática do Braille, bem como na promoção da inclusão e autonomia de pessoas com deficiência visual. Também será investigado como esse impacto pode ser avaliado e quantificado.

\section{Cronograma}

\begin{table}[H]
    \caption{Cronograma de conclusão.}
    \label{tab:treatments}
    \centering
    \begin{tabular}{p{4.3cm} *{5}{p{1.6cm}}}
        \toprule
        \tabhead{Etapa} & \tabhead{Fev'/24} & \tabhead{Mar/24} & \tabhead{Abr/24} & \tabhead{Mai/24} & \tabhead{Jun/24} \\
        \midrule
        Analisar necessidades e requisitos dos usuários             &X&X& & &\\
        \addlinespace
        Projetar e implementar a interface da aplicação             & &X& & &\\
        \addlinespace
        Implementar o Modo Livre                                    & & &X& & \\
        \addlinespace
        Desenvolver o Modo de Lições                                & & &X&X& \\
        \addlinespace
        Realizar testes e avaliações da aplicação com usuários      & & &X&X&X\\
        \bottomrule
    \end{tabular}
\end{table}
