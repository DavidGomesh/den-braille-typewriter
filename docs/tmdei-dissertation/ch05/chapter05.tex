\chapter{Análise de Requisitos} 
\label{chap:Chapter05}

O objetivo deste capítulo é identificar e definir os requisitos funcionais e não funcionais necessários para o desenvolvimento da aplicação. Essa análise visa proporcionar uma solução acessível, prática e eficaz para o aprendizado e prática do Sistema Braille. A análise de requisitos é a etapa no processo de desenvolvimento de software que busca compreender as necessidades dos usuários, garantindo que a solução desenvolvida atenda às expectativas e requisitos estabelecidos.

Inicialmente, serão apresentados os requisitos funcionais, que descrevem as funcionalidades e comportamentos esperados da aplicação. Em seguida, será explicado sobre os requisitos não funcionais, que especificam os atributos de qualidade que a aplicação deve possuir, como usabilidade, desempenho e acessibilidade. Por fim, será detalhado o processo de desenho da solução, explicando as decisões tomadas durante o desenvolvimento e contrapondo estas com possíveis alternativas, visando justificar as escolhas realizadas e assegurar que a solução proposta é a mais adequada para atender aos requisitos identificados. 

A análise servirá como base para as etapas posteriores do desenvolvimento, pois guiará o processo de implementação e garantirá que a aplicação final cumpra com os objetivos propostos.

\section{Requisitos Funcionais}

Os \gls{RF} definem as funcionalidades e comportamentos  que a aplicação deve ter para atender às necessidades dos usuários.

\subsection{Simulação das Teclas da Máquina Braille}

\subsubsection{\gls{RF}01: Mapeamento de Teclas do Teclado do Computador para Pontos Braille}

A aplicação deve permitir que os usuários utilizem o teclado do computador para simular a digitação na Máquina Braille. Os pontos Braille devem ser mapeados para teclas específicas do teclado.

\begin{table}[h]
    \caption{Mapeamento de teclas da Máquina Braille}
    \label{tab:Mapeamento-Teclas-Maquina-Braille}
    \centering
    \begin{tabular}{|p{0.5\linewidth}|p{0.5\linewidth}|} \hline 
        \textbf{Tecla na Máquina Braille} & \textbf{Tecla no computador} \\ \hline
        Ponto 1 & Tecla F \\ \hline
        Ponto 2 & Tecla D \\ \hline
        Ponto 3 & Tecla S \\ \hline
        Ponto 4 & Tecla J \\ \hline
        Ponto 5 & Tecla K \\ \hline
        Ponto 6 & Tecla L \\ \hline
        Espaço & Tecla Espaço \\ \hline
        Quebra de linha & Tecla Q \\ \hline
        Recuo/Apagar caractere & Tecla Backspace \\ \hline
    \end{tabular}
\end{table}

\textbf{Critérios de Aceitação:}
\begin{itemize}
    \item O sistema deve reconhecer corretamente cada tecla pressionada e mapeá-la ao ponto Braille correspondente.
    \item Quando as teclas \textbf{F}, \textbf{D}, \textbf{S}, \textbf{J}, \textbf{K} e \textbf{L} forem pressionadas, o sistema deve exibir a célula Braille correspondente.
    \item Ao pressionar a tecla Q, o sistema deve inserir uma quebra de linha.
    \item O sistema deve apagar o último caractere digitado ao pressionar a tecla Backspace.
    \item O sistema deve ignorar qualquer outra tecla não mapeada que tenha sido pressionada pelo usuário.
\end{itemize}

\subsubsection{\gls{RF}02: Entrada de Caracteres e Comandos através de Combinações de Teclas}

A aplicação deve reconhecer quando o usuário pressiona e solta uma ou mais teclas simultaneamente, correspondendo a um caractere ou comando específico do sistema Braille, e exibir o resultado correspondente na tela.

\textbf{Critérios de Aceitação:}
\begin{itemize}
    \item O sistema deve exibir corretamente o caractere Braille quando o usuário pressiona uma combinação de teclas correspondente.
    \item Quando uma combinação de teclas for pressionada simultaneamente, o sistema deve registrar apenas um caractere.
    \item Quando uma combinação de teclas desconhecida for pressionada, o sistema deve apenas ignorar.
    \item O sistema deve ser capaz de diferenciar entre a pressão de uma tecla individual e uma combinação de teclas.
\end{itemize}

\subsection{Modos de Operação da Aplicação}

\subsubsection{\gls{RF}03: Modo Livre}

A aplicação deve oferecer um Modo Livre que permita aos usuários praticar livremente a digitação em Braille, visualizando em tempo real os caracteres inseridos tanto em representações de células Braille quanto em texto tradicional (a tinta).

\textbf{Critérios de Aceitação:}
\begin{itemize}
    \item O sistema deve permitir que os usuários alternem para o Modo Livre a qualquer momento.
    \item No Modo Livre, o sistema deve exibir em tempo real os caracteres digitados, tanto em Braille quanto em texto a tinta.
    \item O sistema deve permitir que o usuário apague e edite texto no Modo Livre.
\end{itemize}

\subsubsection{\gls{RF}04: Modo Lições}

A aplicação deve incluir um Modo de Lições que ofereça exercícios e lições estruturadas, destinadas a facilitar o aprendizado progressivo do Sistema Braille e do uso da Máquina Braille.

\textbf{Critérios de Aceitação:}
\begin{itemize}
    \item O sistema deve apresentar ao usuário uma série de lições no Modo Lições.
    \item Cada lição deve ter um objetivo claro e ser interativa, permitindo que o usuário pratique e receba feedback.
    \item O sistema deve permitir que o usuário avance para a próxima lição após concluir a lição atual.
\end{itemize}

\subsubsection{\gls{RF}05: Feedback e Avaliação no Modo de Lições}

No Modo de Lições, a aplicação deve fornecer feedback imediato ao usuário sobre a correção de suas entradas, indicando erros e acertos, e permitindo a repetição de exercícios conforme necessário para consolidar o aprendizado.

\textbf{Critérios de Aceitação:}
\begin{itemize}
    \item O sistema deve destacar erros e permitir que o usuário corrija suas respostas antes de prosseguir.
    \item O sistema deve registrar a pontuação do usuário ou fornecer um relatório de desempenho ao final de cada lição.
\end{itemize}

\subsection{Feedback Visual e Sonoro}

\subsubsection{\gls{RF}06: Exibição Visual das Entradas}

A aplicação deve exibir visualmente os caracteres digitados pelo usuário em duas formas:

\begin{itemize}
    \item \textbf{Representação em Células Braille:} mostrando graficamente os pontos correspondentes a cada caractere.
    \item \textbf{Texto a Tinta:} exibindo o equivalente em texto tradicional para facilitar a compreensão por usuários menos familiarizados com o Braille.
\end{itemize}

\textbf{Critérios de Aceitação:}
\begin{itemize}
    \item O sistema deve exibir em tempo real as células Braille correspondentes a cada caractere digitado.
    \item O sistema deve exibir uma opção para trocar para a versão em texto a tinta.
    \item A interface deve atualizar automaticamente a visualização conforme novos caracteres são inseridos ou apagados.
\end{itemize}

\subsubsection{\gls{RF}07: Feedback Sonoro das Entradas}

A aplicação deve fornecer feedback sonoro imediato após cada entrada do usuário, anunciando o caractere inserido, para auxiliar no aprendizado e oferecer uma experiência mais imersiva.

\textbf{Critérios de Aceitação:}
\begin{itemize}
    \item O sistema deve reproduzir um som ou voz correspondente ao caractere inserido imediatamente após o usuário pressionar e soltar as teclas.
    \item O feedback sonoro deve ser claro e sem distorções, permitindo fácil compreensão do caractere digitado.
\end{itemize}

\subsection{Navegação e Edição de Texto}

\subsubsection{\gls{RF}08: Navegação pelo Texto}

A aplicação deve permitir que os usuários naveguem pelo texto digitado utilizando as teclas de seta do teclado, possibilitando a revisão e edição de conteúdo conforme necessário.

\textbf{Critérios de Aceitação:}
\begin{itemize}
    \item O sistema deve permitir a navegação linha por linha e caractere por caractere utilizando as teclas de seta.
    \item O cursor de texto deve se mover de forma visualmente perceptível conforme o usuário navega pelo texto.
    \item O sistema deve permitir que o usuário posicione o cursor em qualquer parte do texto para iniciar a edição.
\end{itemize}

\subsubsection{\gls{RF}09: Edição e Correção de Texto}
Os usuários devem poder editar e corrigir o texto digitado utilizando funcionalidades como recuo/apagar caractere (Tecla Backspace) e inserção de novos caracteres em posições específicas do texto.

\textbf{Critérios de Aceitação:}
\begin{itemize}
    \item O sistema deve apagar o caractere anterior ao cursor quando o usuário pressionar a tecla Backspace.
    \item O sistema deve permitir que o usuário insira novos caracteres em qualquer ponto do texto, movendo o conteúdo existente conforme necessário.
\end{itemize}

\section{Requisitos não Funcionais}

Os \gls{RNF} especificam os atributos de qualidade que a aplicação deve possuir, influenciando a experiência do usuário e a eficácia geral da solução. Estes requisitos abordam aspectos como desempenho, usabilidade e acessibilidade, garantindo que a aplicação funcione corretamente.


\subsection{Usabilidade}

\subsubsection{\gls{RNF}01: Interface Intuitiva e Amigável}
A aplicação deve apresentar uma interface de usuário intuitiva e fácil de usar, permitindo que usuários de diferentes níveis de proficiência tecnológica possam utilizá-la sem dificuldades significativas.

\subsection{Acessibilidade}

\subsubsection{\gls{RNF}02: Conformidade com Padrões de Acessibilidade Web}
A aplicação deve estar em conformidade com os padrões de acessibilidade web, como as Diretrizes de Acessibilidade para Conteúdo Web (WCAG), garantindo que pessoas com diferentes tipos de deficiência possam utilizar a aplicação de forma eficaz.

\subsubsection{\gls{RNF}03: Contraste Adequado e Legibilidade}
Os elementos visuais da aplicação devem apresentar contraste adequado e tipografia legível, facilitando a visualização e compreensão do conteúdo por usuários com diferentes capacidades visuais.

\subsection{Desempenho e Eficiência}

\subsubsection{\gls{RNF}04: Tempo de Resposta Rápido}
A aplicação deve apresentar tempos de resposta rápidos, garantindo que as interações do usuário sejam processadas e exibidas quase instantaneamente, proporcionando uma experiência fluida e sem atrasos perceptíveis.

\subsubsection{\gls{RNF}05: Carregamento Rápido da Aplicação}
O tempo de carregamento inicial da aplicação deve ser minimizado, permitindo que os usuários acessem e comecem a utilizar a aplicação rapidamente, mesmo em conexões de internet mais lentas.

\subsection{Segurança e Privacidade}

\subsubsection{\gls{RNF}06: Conformidade com Regulamentações de Privacidade}
A aplicação deve estar em conformidade com regulamentações de privacidade aplicáveis, como o \gls{LGPD}, assegurando o tratamento adequado e legal dos dados pessoais dos usuários.

\subsection{Portabilidade}
\subsubsection{\gls{RNF}07: Independência de Navegador}
A aplicação deve funcionar de forma consistente e sem erros em diversos navegadores web modernos, evitando dependências específicas que possam limitar o acesso ou a funcionalidade em determinados ambientes.

