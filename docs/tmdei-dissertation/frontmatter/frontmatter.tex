% we include the glossary here (frontmatter is included with \input, so this command is as if it was in main.tex)
%All acronyms must be written in this file.
\newacronym{RF}{RF}{Requisitos Funcionais}
\newacronym{RNF}{RNF}{Requisitos Não Funcionais}
\newacronym{LGPD}{LGPD}{Lei Geral de Proteção de Dados}
\newacronym{IFMA-MTC}{IFMA-MTC}{Instituto Federal de Educação, Ciência e Tecnologia do Maranhão - Campus São Luís - Monte Castelo}

\frontmatter % Use roman page numbering style (i, ii, iii, iv...) for the pre-content pages

\pagestyle{plain} % Default to the plain heading style until the thesis style is called for the body content

%----------------------------------------------------------------------------------------
%	TITLE PAGE
%----------------------------------------------------------------------------------------

\maketitlepage


%----------------------------------------------------------------------------------------
%	STATEMENT of INTEGRITY
%----------------------------------------------------------------------------------------
\integritystatement

%----------------------------------------------------------------------------------------
%	DEDICATION  (optional)
%----------------------------------------------------------------------------------------
%
%\dedicatory{For/Dedicated to/To my\ldots}

%\begin{dedicatory}
%The dedicatory is optional. Below is an example of a humorous dedication.

%"To my wife Marganit and my children Ella Rose and Daniel Adam without whom this book would have been completed two years earlier." in "An Introduction To Algebraic Topology" by Joseph J. Rotman.
%\end{dedicatory}

%----------------------------------------------------------------------------------------
%	ABSTRACT PAGE
%----------------------------------------------------------------------------------------

\begin{abstract}

% here you put the abstract in the main language of the work.

O acesso à Máquina de escrever em Braille e o seu aprendizado tem sido desafiador dado aos altos valores das máquinas disponíveis à venda. Este trabalho apresenta o desenvolvimento de uma aplicação web que simula a Máquina Braille, a fim de proporcionar uma solução acessível para a prática e o aprendizado de Braille e da Máquina Braille. Para isso, a aplicação simula as teclas da máquina no teclado do computador, para que o usuário possa ter uma experiência que facilite a transição para o uso real da Máquina Braille.

A aplicação tem dois modos: o Modo Livre, onde permite aos usuários usarem livremente o simulador e visualizar os textos escritos em células Braille, e o Modo de Lições, onde é oferecido vários exercícios e lições visando facilitar o aprendizado do Sistema Braille e da Máquina Braille. Além disso, a ferramenta inclui a funcionalidade de visualizar o texto a tinta, facilitando usuários menos familiarizados com o Braille.

Este estudo contribui para a acessibilidade e inclusão de pessoas com deficiência visual, principalmente aquelas que adquiriram a deficiência recentemente, pois oferece uma solução prática e eficaz para o aprendizado do Braille. Permitir que os usuários pratiquem a Máquina Braille de forma virtual e interativa é facilitar a transição para o uso real da Máquina Braille, assim, promovendo autonomia e independência das pessoas com deficiência visual.

\end{abstract}

\begin{abstractotherlanguage}
% here you put the abstract in the "other language": English, if the work is written in Portuguese; Portuguese, if the work is written in English.

Access to the Braille typewriter and learning it has been challenging due to the high prices of the machines available for sale. This work presents the development of a web application that simulates the Braille Machine, aiming to provide an affordable solution for practicing and learning Braille and the Braille Machine. The application simulates the machine's keys on the computer keyboard, allowing the user to have an experience that facilitates the transition to using the Braille Machine in real life.

The application has two modes: Free Mode, which allows users to freely use the simulator and view the texts written in Braille cells, and Lesson Mode, which offers various exercises and lessons to facilitate learning of the Braille System and the Braille Machine. Additionally, the tool includes the functionality to view the text in ink, making it easier for users less familiar with Braille.

This study contributes to the accessibility and inclusion of people with visual impairments, especially those who have recently acquired the impairment, as it offers a practical and effective solution for learning Braille. Allowing users to practice the Braille Machine virtually and interactively facilitates the transition to using the Braille Machine in real life, thus promoting autonomy and independence for people with visual impairments.


\end{abstractotherlanguage}

%----------------------------------------------------------------------------------------
%	ACKNOWLEDGEMENTS (optional)
%----------------------------------------------------------------------------------------

%\begin{acknowledgements}

%The optional Acknowledgment goes here\ldots Below is an example of a humorous acknowledgment.

%"I'd also like to thank the Van Allen belts for protecting us from the harmful solar wind, and the earth for being just the right distance from the sun for being conducive to life, and for the ability for water atoms to clump so efficiently, for pretty much the same reason. Finally, I'd like to thank every single one of my forebears for surviving long enough in this hostile world to procreate. Without any one of you, this book would not have been possible." in "The Woman Who Died a Lot" by Jasper Fforde.
%\end{acknowledgements}

%----------------------------------------------------------------------------------------
%	LIST OF CONTENTS/FIGURES/TABLES PAGES
%----------------------------------------------------------------------------------------

\tableofcontents % Prints the main table of contents

\listoffigures % Prints the list of figures

\listoftables % Prints the list of tables

\iflanguage{portuguese}{
\renewcommand{\listalgorithmname}{Lista de Algor\'itmos}
}
\listofalgorithms % Prints the list of algorithms
\addchaptertocentry{\listalgorithmname}


\renewcommand{\lstlistlistingname}{List of Source Code}
\iflanguage{portuguese}{
\renewcommand{\lstlistlistingname}{Lista de C\'odigo}
}
\lstlistoflistings % Prints the list of listings (programming language source code)
\addchaptertocentry{\lstlistlistingname}


%----------------------------------------------------------------------------------------
%	ABBREVIATIONS
%----------------------------------------------------------------------------------------
%\begin{abbreviations}{ll} % Include a list of abbreviations (a table of two columns)
%%\textbf{LAH} & \textbf{L}ist \textbf{A}bbreviations \textbf{H}ere\\
%%\textbf{WSF} & \textbf{W}hat (it) \textbf{S}tands \textbf{F}or\\
%\end{abbreviations}

%----------------------------------------------------------------------------------------
%	SYMBOLS
%----------------------------------------------------------------------------------------

\begin{symbols}{lll} % Include a list of Symbols (a three column table)

%$a$ & distance & \si{\meter} \\
%$P$ & power & \si{\watt} (\si{\joule\per\second}) \\
%Symbol & Name & Unit \\

%\addlinespace % Gap to separate the Roman symbols from the Greek

%$\omega$ & angular frequency & \si{\radian} \\

\end{symbols}



%----------------------------------------------------------------------------------------
%	ACRONYMS
%----------------------------------------------------------------------------------------

\newcommand{\listacronymname}{List of Acronyms}
\iflanguage{portuguese}{
\renewcommand{\listacronymname}{Lista de Acr\'onimos}
}

%Use GLS
\glsresetall
\printglossary[title=\listacronymname,type=\acronymtype,style=long]

%----------------------------------------------------------------------------------------
%	DONE
%----------------------------------------------------------------------------------------

\mainmatter % Begin numeric (1,2,3...) page numbering
\pagestyle{thesis} % Return the page headers back to the "thesis" style
